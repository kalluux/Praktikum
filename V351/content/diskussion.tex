\section{Diskussion}
\label{sec:Diskussion}

Die berechneten Steigungen betragen für die Rechteckpannung $m_R = -1.35 \pm 0.07$, für die
Dreieckspannung $m_D= -2.35 \pm 0.17$ und für die Sägezahnspannung $m_S= -1.01 \pm 0.07$. 
Die Steigungen $m_R$ und $m_S$ sollten $-1$ betragen, und weichen somit um 35 Prozent bzw. 1 Prozent ab. 
Die Steigung $m_D$ sollte $-2$ betragen, und weicht um 17,5 Prozent von dem Wert ab. Die Abweichung 
der Dreieckspannung kann sich dadurch erklären lassen, dass sie um $\frac{1}{n^2}$ abfällt, und es unmöglich war, mehr als 8 Amplituden zu messen.
Die Messung der Amplituden der Sägezahnspannung lief am genausten. 
Insgesamt entspricht der Abfall der Ampliuden aber den theoretischen Werten.

\noindent Die Synthese der Dreieck- und Sägezahnspannung konnte nicht genau stattfinden. 
Die Amplituden sind vor allem bei der Dreieckspannung zu weit gefallen, und somit konnten nur 2 Oberwellen eingestellt werden, da sich am Oberwellengenerator keine niedrigeren Spannungen als
$0.039$ einstellen ließ.
Somit ist der Oberwellengenerator als Fehlerquelle zu nennen.