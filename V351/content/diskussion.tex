\section{Diskussion}
\label{sec:Diskussion}

Die berechneten Steigungen betragen
\begin{align}
m_R &= -1,08 \pm 0,06 \\
m_D &= -1,84 \pm 0,09 \\
m_S &= -1,01 \pm 0,07 .
\end{align}


Die Steigungen $m_R$ und $m_S$ sollten $-1$ betragen, und weichen somit um $8$ Prozent bzw. $1$ Prozent ab. 
Die Steigung $m_D$ sollte $-2$ betragen, und weicht um $8$ Prozent von dem Wert ab. Die Abweichung 
der Dreieckspannung kann sich dadurch erklären lassen, dass sie um $\frac{1}{n^2}$ abfällt, und es unmöglich war, mehr als 8 Amplituden zu messen.
Die Messung der Amplituden der Sägezahnspannung lief am genausten. 
Insgesamt entspricht der Abfall der Amplituden aber in etwa den theoretischen Werten, obwohl das Ablesen der Werte am Oszilloskop nur ungenau vorgenommen werden konnte, da dies ohne 
Hilfsmittel stattfinden konnte und die Amplituden nicht konstant auf einem Wert standen.

\noindent Die Synthese der Dreieck- und Sägezahnspannung konnte nicht genau stattfinden, obwohl die Formen der Spannungen mit kleinen Abweichungen auf den Abbildungen zu erkennen sind.
Die Amplituden sind vor allem bei der Dreieckspannung zu weit gefallen, und somit konnten nur 2 Oberwellen eingestellt werden, da sich am Oberwellengenerator keine niedrigeren Spannungen als
$0,039$ $\si{volt}$ einstellen ließ.
Somit ist der Oberwellengenerator als Fehlerquelle zu nennen. 