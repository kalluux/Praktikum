\section{Auswertung}
\label{sec:Auswertung}

\subsection{Betimmung der Winkelrichtgröße}
Die Winkelrichtgröße wird aus Gleichung (4) bestimmt. In Tabelle () sind alle dafür benötigten Größen, also 
Abstand des Mittelpunktes der Drillachse zur angesetzten Federwaage, Auslenkungswinkel, und gemessene Kraft, dargestellt, 
wie zuletzt auch die berechnete Winkelrichtgröße. Dabei wurde für je einen Winkel zweimal der Abstand geändert und
gemessen. Die Unsicherheit des Abstandes wird auf 0.05 mm geschätzt.


\begin{table}[H]
  \centering
  \caption{Werte zur Berechnung der Winkelrichtgröße}
  \label{tab:Parameter}
  \begin{tabular}{c c c c}
    \toprule
    $\phi$ & $F/$N & r/m & $D/$Nm/$10^{-3}$\\
    \bottomrule
     30° & 0.46  & $0.02965 \pm 0.00005$ & $0.455 \pm 0.00077$ \\
     30° & 0.26  & $0.04945 \pm 0.00005$ & $0.429 \pm 0.00043$\\
     40° & 0.62  & $0.02965 \pm 0.00005$ & $0.460 \pm 0.00076$\\
     40° & 0.39  & $0.04945 \pm 0.00005$ & $0.482 \pm 0.00049$\\
     50° & 0.80  & $0.02965 \pm 0.00005$ & $0.474 \pm 0.00080$\\
     50° & 0.48  & $0.04945 \pm 0.00005$ & $0.475 \pm 0.00048$\\
     60° & 0.94  & $0.02965 \pm 0.00005$ & $0.465 \pm 0.00078$\\
     60° & 0.57  & $0.04945 \pm 0.00005$ & $0.470 \pm 0.00048$\\
     70° & 1.10  & $0.02965 \pm 0.00005$ & $0.466 \pm 0.00079$\\
     70° & 0.66  & $0.04945 \pm 0.00005$ & $0.466 \pm 0.00047$\\
    \bottomrule
  \end{tabular}
\end{table}

Der Mittelwert aller Winkelrichtgrößen beträgt:

\begin{equation}
  D = (0.464 \pm 0.0005)\cdot 10^{-3} \symup{Nm}
\end{equation}






\subsection{Bestimmung des Trägheitsmoments der Drillachse}
In folgender Tabelle wird die Schwingungsdauer T und der zugehörige Abstand a vom Mittelpunk der Drillachse bis 
zum Schwerpunkt der Gewichte dargestellt. Dabei wurde für einen Abstand 10 Schwingungen gemessen und das Ergebnis
durch 10 geteilt.


\begin{table}[H]
  \centering
  \caption{Gemessene Schwingungsdauern und Abstände}
  \label{tab:Gemessene Schwingungsdauern und Abstände}
  \begin{tabular}{c c}
    \toprule
    $a/$mm & $T/$s \\
    \midrule
     $60 \pm 0.05$ & $2.46 \pm 0.5$ \\ 
     $80 \pm 0.05$ & $2.79 \pm 0.5$ \\
    $100 \pm 0.05$ & $3.16 \pm 0.5$ \\
    $120 \pm 0.05$ & $3.54 \pm 0.5$ \\
    $140 \pm 0.05$ & $3.97 \pm 0.5$ \\
    $160 \pm 0.05$ & $4.40 \pm 0.5$ \\
    $180 \pm 0.05$ & $4.86 \pm 0.5$ \\
    $200 \pm 0.05$ & $5.28 \pm 0.5$ \\
    $220 \pm 0.05$ & $5.76 \pm 0.5$ \\
    $240 \pm 0.05$ & $6.24 \pm 0.5$ \\
    \bottomrule
  \end{tabular}
\end{table} 







\subsection{Bestimmung des Trägheitsmoments für zwei Körper}
In diesem Auswertungsteil werden die Trägheitsmomente für eine Kugel und einen 
Zylinder berechnet, deren Drehachsen ihren Symmetrieachsen entsprechen.
Für die Kugel wurden je 8 Schwingungen gemessen, für den Zylinder je 5.
\begin{table}[H]
  \centering
  \caption{Schwingungsdauer eines Zylinder und einer Kugel}
  \label{tab:Schwingungsdauer von Zylinder und Kugel}
  \begin{tabular}{c c}
    \toprule
    $T_Z/$s & $T_K/$s \\
    \midrule
    0.72 & 1.47 \\
    0.75 & 1.45 \\
    0.74 & 1.48 \\
    0.74 & 1.44 \\
    0.74 & 1.46 \\
    \bottomrule
  \end{tabular}
\end{table}




\subsection{Bestimmung des Trägheitsmoments einer Modellpuppe}
Es wird die Schwingungsdauer einer Puppe für zwei unterschiedliche 
Posen $P_1$ und $P_2$ bestimmt. Bei der ersten Pose sind Arme und Beine am Körper angewinkelt und in der zweiten
sind Arme senkrecht zum Körper nach außen gestreckt, und die Beine entgegengesetzt nach hinten bzw. vorne gestreckt.
Für die erste Pose werden 5 Schwingungen gemessen, für die zweite 10.
\begin{table}[H]
  \centering
  \caption{Schwingungsdauer der Modellpuppe}
  \label{tab:Schwingungsdauer der Modellpuppe}
  \begin{tabular}{c c}
    \toprule
    $T_{P_1}/$s & $T_{P_2}/$s \\
    \midrule
    $0.34 \pm 0.5$ & $0.85 \pm 0.5$ \\
    $0.35 \pm 0.5$ & $0.85 \pm 0.5$ \\
    $0.36 \pm 0.5$ & $0.84 \pm 0.5$ \\
    $0.38 \pm 0.5$ & $0.85 \pm 0.5$ \\
    $0.35 \pm 0.5$ & $0.86 \pm 0.5$ \\
    \bottomrule
  \end{tabular}
\end{table}




