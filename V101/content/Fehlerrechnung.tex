\section{Fehlerrechnung}
\label{sec:Fehlerrechnung}
Der Mittelwert berechnet sich mit folgender Formel:
\begin{equation}
  \bar{x} = \frac{1}{N} \sum_{i=1}^N x_i
\end{equation}
Der Fehler des Mittelwertes lautet entsprechend :
\begin{equation}
  \sigma = \sqrt{\frac{1}{N(N-1)} \sum_{i=1}^N (x_i - \bar{x})}
\end{equation}
Werden Daten mit Unsicherheiten in späteren Formeln weiter verwendet, breiten sich die Fehler nach der Gauß'schen Fehlerfortpflanzung aus:
\begin{equation}
  \sigma_f = \sqrt{
      \sum\limits_{i = 1}^N
       \left( \frac{\partial f}{\partial x_i} \sigma_i \right)^{\!\! 2}
     }
\end{equation}
Ausgleichsrechnung wird mit folgender Formel durchgeführt :
\begin{align}
  y = a \cdot x + b \\
  a = \frac{\overline{xy}-\bar{x}\cdot\bar{y}}{\bar{x^2}-\bar{x}^2} \\
  b = \frac{\bar{x^2}\bar{y}-\overline{xy}\bar{x}}{\bar{x^2}-\bar{x}^2}
\end{align}

