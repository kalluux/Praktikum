\section{Diskussion}
\label{sec:Diskussion}

Die Abstände der Gewichte auf der masselosen Stange konnten nur
ungenau bestimmt werden, was gemeinsam mit der prinzipiell fehlerbehafteten ungenauen Zeitmessung möglicherweise das negative Trägheitsmoment
der Drehachse hervorbrachte. Auch der Einsatz einer masselosen Stange ist nicht
zu verwirklichen, wenn auch die Formel für die Berechnung des Trägheitsmoments dieser gegeben und
in Abbildung 1 dargestellt ist, sodass dieses in der Berechnung hätte berücksichtigt werden können.
Dies hätte vermutlich deutlich realistischere Werte ergeben, da diese etwa 97\,\si{\gram} wiegt und eine große radiale Ausdehnung hat, nach Steiner
also ein großes Trägheitsmoment aufweist.
Das genaue Messen mit der Stoppuhr ist durch die menschliche Reaktionszeit für
kurze Schwingungsdauern, wie in Position 1 fast unmöglich genau zu realisieren, weshalb
unter Umständen von großen Abweichungen der Zeit ausgegangen werden muss.
Bei der Schwingungsdauer wird in Gleichung (6) von kleinen Winkel $\phi$ ausgegangen, was für gewisse Messungen
nicht beachtbar war. Diese Kleinwinkelnäherung ist somit stark fehlerbehaftet, womit sich in allen Rechnungen mit der Schwingungsdauer
T ein systematischer Fehler ergibt. Unter anderem deshalb unterscheiden sich
die Trägheitsmomente der Kugel bzw. des Zylinders von den Theoriewerten um
ca. 98 Prozent. Der Zylinder hat außerdem eine sehr kurze Schwingungsdauer,
weshalb bei der Durchführung nur 5 Schwingungen gemessen wurden und für die Kugel 8, sodass auch auf dieser Zeitmessung ein relativ großer Fehler lastet.
Desweiteren konnte die Drillachse nicht in die Berechnung mit einbezogen werden, was realitätsfern ist.
Auch die Schwingungsdauer der Puppe war aufgrund des geringen Trägheitmomentes sehr kurz, weshalb 
für die erste Position nur 5 Schwingungen gemessen wurden.
Die systematischen Fehler sind die gleichen wie bei den zwei Körpern.
Die relative Abweichung der Trägheitsmomente zu den Theoriewerten beträgt 96,34 beziehungsweise 98,40 Prozent. 
Für die Berechnung des Trägheitsmoments wurden auch Vereinfachungen angenommen,
denn die Puppe wurde nicht genau in alle ihre Einzelteile zerlegt, sondern wurde nur angenähert,
weshalb sich der Steiner'sche Satz nicht exakt verifizieren ließ. Dennoch zeigen die berechneten Werte, dadurch
dass das Trägheitsmoment der zweiten Position deutlich höher ist, einen Trend in die richtige Richtung.