\section{Diskussion}
\label{sec:Diskussion}

Die Abstände der Gewichte auf der masselosen Stange konnten nur
ungenau bestimmt werden, was gemeinsam mit der ungenauen Zeitmessung möglicherweise das negative Trägheitsmoment
der Drehachse hervorbrachte. Auch der Einsatz einer masselosen Stange ist nicht
zu verwirklichen.
Das genaue Messen mit der Stoppuhr ist fast unmöglich genau zu realisieren, aufgrund der 
kurzen Schwingungsdauer, weshalb von großen Abweichungen der Zeit ausgegangen werden muss.
Bei der Schwingungsdauer wird in Gleichung (6) von kleinen Winkel $\phi$ ausgegangen, was für gewisse Messungen
nicht beachtbar war. Ab $\phi$ \(\gg\) 5° ist diese
Kleinwinkelnäherung somit stark fehlerbehaftet, womit sich in allen Rechnungen mit der Schwingungsdauer
T ein systematischer Fehler ergibt. Unter anderem deshalb unterscheiden sich
die Trägheitsmomente der Kugel bzw. des Zylinders von den Theoriewerten um
ca. 98 Prozent. Der Zylinder hat außerdem eine sehr kurze Schwingungsdauer,
weshalb bei der Durchführung 5 Schwingungen gemessen wurden und für die Kugel 8.
Außerdem konnte die Drillachse nicht in die Berechnung mit einbezogen werden, was realitätsfern ist.
Auch die Schwingungsdauer der Puppe war aufgrund des geringen Trägheitmomentes sehr kurz, weshalb 
für die erste Position nur 5 Schwingungen gemessen wurden.
Die systematischen Fehler sind die gleichen wie bei den zwei Körpern.
Die relative Abweichung der Trägheitsmomente zu den Theoriewerten beträgt 96,34 beziehungsweise 98,40 Prozent. 
Für die Berechnung des Trägheitsmoments wurden auch starke Vereinfachungen angenommen,
denn die Puppe wurde nicht genau in alle ihre Einzelteile zerlegt sondern es wurde relativ viel angenähert,
weshalb sich der Steiner'sche Satz nicht verifizieren konnte. Dennoch zeigen die berechneten Werte, dadurch
dass das Trägheitsmoment der zweiten Position deutlich höher ist, einen Trend in die richtige Richtung.