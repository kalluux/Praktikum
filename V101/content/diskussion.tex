\section{Diskussion}
\label{sec:Diskussion}

Die Abstände der Gewichte auf der masselosen Stange konnten nur
sehr ungenau bestimmt werden, weshalb möglicherweise das negative Trägheitsmoment
der Drehachse zustande kommt. Auch der Einsatz einer masselosen Stange ist nicht
zu verwirklichen. 
Laut Versuchsanleitung gilt Gleichung (6) nur für kleine Winkel, was 
bei diesem Versuch nicht möglich war zu beachten. Deshalb unterschieden sich
die Trägheitsmomente der Kugel bzw. des Zylinders zu den Theoriewerten um
ca. 98 Prozent. Der Zylinder hat außerdem eine sehr kurze Schwingungsdauer,
weshalb bei der Durchführung 5 Schwingungen gemessen wurden und für die Kugel 8.
Das genaue Messen mit der Stoppuhr ist fast unmöglich genau zu realisieren, auf Grund der 
kurzen Schwingungsdauer, deshalb muss von großen Abweichungen der Zeit ausgegangen werden.
Außerdem konnte die Drillachse nicht mit einbezogen werden, was falsch ist.
Auch die Schwingungsdauer der Puppe war sehr kurz, weshalb für die erste Pose nur 5 Schwingungen gemessen wurden.
Die systematischen Fehler sind die gleichen wie bei den zwei Körpern.
Die Abweichung der Trägheitsmomente zu den Theoriewerten beträgt: ... bzw. ...
Für die Berechnung des Trägheitsmomentes wurden auch starke Vereinfachungen angenommen,
denn die Puppe wurde nicht komplett in alle ihre Einzelteile zerlegt. Nichts desto trotz ...
