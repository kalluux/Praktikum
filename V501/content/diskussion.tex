\section{Diskussion}
\label{sec:Diskussion}

Die prozentuale Abweichung des Wertes für $\frac{pL}{2d} = 0,3575 \si{\meter} $ und dem berechneten $a = (0,324 \pm 0,015)  \si{\meter}$, liegt bei ca. $9$ Prozent.
Dies lässt sich unter anderem dadurch erklären, dass die Ablenkung auf dem Koordinatensystem an der Röhre nicht präzise abgelesen werden konnte.
Die Messwerte liegen sehr nah an den Ausgleichsgeraden.

\noindent Bei der Messung der Frequenz kann zu keinem Zeitpunkt ein stehendes Bild erzeugt werden, somit ist eine genaue Bestimmung unmöglich. 
Trotzdem passen die gemessenen Frequenzen für die unterschiedlichen $n$ gut überein.

\noindent 
Der ermittelte Wert für die spezifische Elektronenladung beträgt 
\begin{align*}
\frac{e_0}{m_0} = (1.72 \pm 0.06) \cdot 10^{11} \symup{\frac{C}{kg}} .
\end{align*}
Vergleicht man diesen mit dem Literaturwert (\cite{kent3})
\begin{align*}
\frac{e_0}{m_0}_{lit} = 1.76 \cdot 10^{11} \symup{\frac{C}{kg}},
\end{align*}
ergibt sich eine Abweichung von ca. $2,3$ Prozent. Trotz relativ geringer Abweichung, gab es einige Fehlerquellen.

Die Totalintensität des Erdmagnetfeldes betrug im Versuch 
\begin{align*}
B_{total} = B \cdot sin(\phi) = 9,421 \cdot 10^{-6} \si{\tesla},
\end{align*}
und ein Theoriewert in Mitteleuropa\cite{kent4} ist etwa
\begin{align*}
B_{total,theorie} =  4,9 \cdot 10^{-5} \si{\tesla},
\end{align*}
womit es eine relative Abweichung von 80,77\% gibt.
Eine primäre Fehlerquelle ist der Kompass, da sich dieser nie richtig auslenken konnte.
Somit konnte die Röhre nicht in die richtige Position gebracht werden. 
Auch der Inklinationswinkel konnte deshalb nur sehr ungenau bestimmt werden, weshalb die Bestimmung der Totalintensität des Erdmagnetfeldes fehlerhaft war.