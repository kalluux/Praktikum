\section{Diskussion}
\label{sec:Diskussion}

% 0,447 , 0,443 , 0,453

Bei dem Vergleich der Ergebnisse fällt auf, dass die errechneten magnetischen Momente 
verhältnismäßig nah beieinander liegen, wobei alle aufgrund unterschiedlicher Gründe 
fehlerbehaftet sind. Bei der Bestimmung über die Gravitation ist es mit dem bloßen Auge nicht möglich, 
den genauen Zeitpunkt zu bestimmen, bei der die auf den Stab wirkenden Kräfte nicht mehr im 
Gleichgewicht sind und der Stab zu kippen beginnt. Ausschlaggebend für die Fehler bei der Schwingungsdauermethode 
ist vorallem die menschliche Reaktionszeit beim Stoppen der Zeit, welche durch mehrfache Messung relativiert 
werden könnte. Allerdings bleibt es schwierig, den Stab immer wieder gleich stark auszulenken.
Am meisten fehleranfällig ist die Methode, bei der die Präzession ausgenutzt wird, da es aufgrund der aufwendigen 
Durchführung insgesamt zuviele Variablen gibt, wie beispielsweise die nicht konstante Rotation, die 
Reaktionszeit beim Stoppen der Zeit und das manuelle Andrehen und Auslenken.

\noindent Ein Vergleich mit Literaturwerten ist in diesem Versuch nicht möglich.