\section{Diskussion}
\label{sec:Diskussion}
Die Bildschirmfotos des Oszilloskops zeigen, dass sich der Spannungsverlauf mit wachsender
Phasenverschiebung deutlich verändert. Nach einer Phasenverschiebung von  $\varphi = \pi$
wiederholt sich das Bild in etwa, nur dass das Vorzeichen umgekehrt wird.
Die tatsächlich angelegte Eingangsspannung $U_\text{0}$
beträgt während der Messreihen 6,45\,V, sodass die relative Abweichung der berechneten
Eingangsspannungen 1,71\% und 2,64\% beträgt. Ein Vergleich der Ergebnisse der ersten beiden Messreihen zeigt, dass ein 
Lock-In-Verstärker ideal für die Unterdrückung von Störsignalen geeignet ist und in der Lage ist,
die gewünschte Spannung zu verstärken.

Die Messwerte der Lichtintensität zur Überprüfung der Rauschunterdrückung mittels LED und Photodetektor sinken
zwar kontinuierlich, aber in nicht in dem Maße in dem sie es sollten. Auf Grund des Abstandsgesetzes sollte die Steigung $a = -2$ betragen.
Der berechnete Wert lautet $a = -2,54$, welcher um 27\% vom Theoriewert abweicht. Die Vermutung liegt nahe,
dass das Licht aus der Umgebung einen zu großen Einfluss auf das Experiment hat, welchem durch
Abdunklung oder ähnlichem entgegen gewirkt werden könnte. Dies kann auch der Grund für die teils
stark schwankenden Intensitäten sein, da alle Werte mit dem variablen Hintergrundrauschen belastet sind.
Gerade ab Abständen von mehr als 80\,cm überwiegt das Rauschen deutlich, sodass weitere Messungen
nicht zielführend sind und abgebrochen wurden.