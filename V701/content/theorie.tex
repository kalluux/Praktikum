\section{Zielsetzung}
\label{sec:Zielsetzung}
In diesem Versuch sollen die Reichweite und der Energieverlust von Alphastrahlung in Luft
bestimmt werden. Außerdem soll die Statistik des radioaktiven
Zerfalls untersucht werden.

\section{Theorie}
\label{sec:Theorie}
Eine experimentelle Bestimmung der Energie von Alphastrahlung ist durch Messung
ihrer Reichweite möglich. Alphateilchen verlieren beim Durchlaufen eines Mediums ihre Energie
hauptsächlich durch Ionistationsprozesse, sowie durch Anregung oder Dissoziation von Molekülen.
Elatische Stöße spielen hierbei nur eine untergeordnete Rolle. Der Energieverlust $-\text{d}E_\alpha / \text{d}x$
hängt dabei von der Energie der Strahlung und der Dichte des Mediums statt und lässt sich
für hinreichend große Energien durch die Bethe-Bloch-Gleichung
\begin{align}
  - \frac{\text{d}E_\alpha}{\text{d}x} = \frac{z^2 e^4}{4\pi \epsilon_0^{2} m_\text{e}} \frac{n Z}{v^2} \ln{\left(\frac{2m_\text{e}v^2}{I}\right)}
\end{align}
beschreiben, wobei $z$ die Ladung und $v$ die Geschwindigkeit der Alphateilchen darstellt.
Die Ordnungszahl sei $Z$, $n$ ist die Teilchendichte und $I$ die Ionisierungsenergie
des Targetgases. Die Bethe-Bloch-Gleichung verliert durch Auftreten von Ladungsaustauschprozessen
für sehr kleine Energien ihre Gültigkeit. \\
Die Reichweite $R$ eines Alphateilchens bezeichnet seine Wegstrecke bis zur vollständigen Abbremsung
und lässt sich durch
\begin{align}
  R = \int_0^{E_0} \frac{\text{d}E_{\alpha}}{-\text{d}E_{\alpha} / \text{d}x}\,
\end{align}
bestimmen. Bevor ein Alphateilchen vollständig abgebremst ist, verbringt es zwingend eine Zeit 
mit niedriger Energie, in der die Bethe-Bloch-Gleichung nicht aussagekräftig ist.
Deswegen werden zur Bestimmung der mittleren Reichweite $R_{\text{m}}$ empirische Formeln
verwendet. So gilt zum Beispiel näherungsweise für die Reichweite von Alphastrahlung mit einer
Energie $E_\text{\alpha} \leq 2,5\,\si{\mega\eV}$ in Luft
\begin{align}
  R_{\text{m}} = 3{,}1 \cdot E_\text{\alpha}^{3/2}\,,
\end{align}
wobei $R_{\text{m}}$ in Millimetern und $E_\text{\alpha}$ in Megaelektronenvolt anzugeben ist.\\
Für die Reichweite von Alphateilchen in Gasen gilt, dass sie bei konstanter Temperatur
und konstantem Volumen proportional zum Druck $p$ ist. Es gilt für einen festen Abstand $x_\text{0}$
zwischen Detektor und Quelle der Alphastrahlung
\begin{equation}
  x = x_\text{0} \frac{p}{p_\text{0}}\,,
\end{equation}
wobei $x$ die effektive Weglänge bezeichnet und $p_\text{0} = 1013\,\si{\milli\bar}$ der Normaldruck ist.