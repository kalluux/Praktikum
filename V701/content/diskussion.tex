\section{Diskussion}
\label{sec:Diskussion}

Die bestimmten Werte für die Reichweiten und die Energien
\begin{align*}
E_{1} &= (2,21 \pm 0,1) \si{\MeV}\\
E_{2} &= (1,8 \pm 0,19) \si{\MeV}\\
R_{m1} &= (1,02 \pm 0,07) \si{\centi\meter}\\
R_{m2} &= (0,75 \pm 0,12) \si{\centi\meter}\\
\end{align*}
weichen leicht voneinander ab, obwohl die Ergebnisse gleich für die verwendeten Abstände sein sollten.
Die Abweichungen der gesamten Messung lassen sich dadurch erklären, dass keine vollständig linearen Messungen durchgeführt wurden,
und sich somit nur für einen Teil,also für wenige Messwerte, eine Linearität hat erkennen lassen. Die Parameter waren jedoch Grundlage für die berechneten Größen.
Außerdem liefen auf Grund der eingestellten Diskriminatorschwelle gegen Ende der Messung die Kanal auf den gleichen Wert hinaus.
Gleiches lässt sich für die Berechnung des Energieverlustes sagen. Die damit berechneten Energien
\begin{align*}
E_1 &= (2,91 \pm 0,13) \si{\MeV}\\
E_2 &= (2,91 \pm 0,17) \si{\MeV}
\end{align*}
sind bis auf den Fehler identisch; weichen aber auch von den oben berechneten Energien ab, vor allem, weil, wie oben gesagt, nicht alle Werte für die Ausgleichsrechnung verwendet werden konnten. 
\noindent Für den radioaktiven Zerfall wird theoretisch eine Poisson-Verteilung erwartet. Aus dem Histogramm lässt sich schwer abschätzen, welcher Verteilung diese Messung ähnelt,
da aber bei einer Poisson-Verteilung der Mittelwert und die Varianz nahezu gleich sein sollten, und dies hier nicht der Fall ist, ähnelt diese Messung eher einer Gauß-Verteilung. 
Außerdem fällt auf, dass das Quadrat der Standardabweichung ungefähr der Varianz entspricht. Dies ist bei einer Gauß-Verteilung üblich.
