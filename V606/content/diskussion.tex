\section{Diskussion}
\label{sec:Diskussion}

In Tabelle 6 werden die experimentell bestimmten Suszeptibilitäten mit den theoretisch berechneten Suszeptibilitäten verglichen.
\begin{table}[H]
  \centering
  \caption{Vergleiche der Suszeptibilitäten.}
  \label{tab:Dy}
  \begin{tabular}{c c c c c c}
    \toprule
    Probe & $\chi_T$ & $\chi_U$ & $\chi_R$ & $\frac{\chi_T - \chi_U}{\chi_T}/$\%  & $\frac{\chi_T - \chi_R}{\chi_T}/$\% \\
    \midrule
    Dy & 0,0251 & 0,00566 \pm 0,00006 & 0,01961 \pm 0,00024 &  77,45 & 21,87  \\  
    Nd & 0,00302 & 0,00015 \pm 0,00005 & 0,00226 \pm 0,00024 &  95,03  & 25,17 \\    
    Pr & 0,00169 & 0,00002 \pm 0,00001 & 0,00101 \pm 0,00014 &  98,82  & 40,24 \\
    Gd & 0,0137 &  0,00258 \pm 0,00010 & 0,00954 \pm 0,00018 &   81,17 & 30,37\\   
    \bottomrule
  \end{tabular}
\end{table}

Die Abweichungen der berechneten Suszeptibilitäten zu den theoretischen Werten sind insgesamt sehr groß. Die Fehler müssen bei der Messung über die Spannung größer gewesen sein, da die Abweichungen hier nahezu 100 Prozent sind.
Eventueller Messfehler ist die nur angenommene Temperatur. Auch die Widerstände konnten nur ungenau eingeregelt werden. Die Längen der Proben wurden nur auf die Spulenlänge genähert.
Das Ablesen der Spannungen war fehlerhaft, da diese geschwankt haben, auch je nach gewählter Skala.
Ein systematischer Fehler ist außerdem, dass die Dichte von Praseodym genommen wurde, und nicht von dem hier verwendeten Oxid.

\noindent Die Filterkurve des Selektivverstärkers entspricht der erwarteten Kurve. Die Güte von $97,5$ liegt nah an dem zu erwarenden Wert, mit nur $2,5$ prozentiger Abweichung.