\section{Auswertung}
\label{sec:Auswertung}

\subsection{Messung ohne Magnetfeld}
Das Gesamtträgheitsmoment $\Theta_\text{ges}$ ergibt sich aus der Summe des Trägheitsmomentes
der Kugel $\Theta_\text{Kugel}$ nach Gleichung (11), und dem gegebenen Trägheitsmoment der Halterung $\Theta_\text{h}$ zu
\begin{align*}
  \Theta_\text{ges} = {} & \Theta_\text{Kugel} + \Theta_\text{h} \\
                    = {} & (1,53 \cdot 10^{-4} \pm 1,4 \cdot 10^{-7})\,\si{\kilo\gram\meter\squared} + 2,25 \cdot 10^{-6}\,\si{\kilo\gram\meter\squared} \\
                    = {} & (1,5525 \cdot 10^{-4} \pm 1,4 \cdot 10^{-7})\,\si{\kilo\gram\meter\squared}.
\end{align*}

\noindent Zur Bestimmung des Schubmoduls $G$ werden nun 10 Schwingungsperioden ohne Magnetfeld $B$ gemessen (Tabelle 1).


\begin{table}[H]
\centering
\caption{Periodendauer $T$ des zur Schwingung angeregten Drahtes ohne Magnetfeld $B$.}
\label{tab:ohnebfeld}
\begin{tabular}{c c}
\toprule
Fehler & T\:/\:s\\
\midrule
 & 18,247 \\
 & 18,242 \\
 & 18,244 \\
 & 18,245 \\
 & 18,250 \\
 & 18,249 \\
 & 18,243 \\
 & 18,235 \\
 & 18,248 \\
 & 18,250 \\
\hline
Mittelwert & 18,245 \\
Standardabweichung & 0,004 \\
\bottomrule
\end{tabular}
\end{table}

\noindent Mit der Länge $L$ und dem Durchmesser $d$ des Drahtes lässt sich nach Gleichung (12) den Schubmodul $G$ zu 

\begin{table}
\centering
\caption{Messgrößen des Drahtes.}
\label{tab:ohnebfeld}
\begin{tabular}{c c c}
\toprule
 & d\:/\:mm & L\:/\:cm\\
\midrule
 & 0,172 & 66,30 \\
 & 0,173 &  \\
 & 0,170 & \\
\hline
Mittelwert & 0,172 & \\
Standardabweichung & 0,001 & \\
\bottomrule
\end{tabular}
\end{table}

\begin{align*}
G = (1,41 \pm 0,04) \cdot 10^{13}\,\si{\newton\per\meter\squared}
\end{align*}
berechnen.

Der Elastizitätsmodul $E$ wird aufgrund der fehlenden experimentellen Bestimmung nach Literaturwerten
\begin{align*}
E = 210\,\si{\giga\pascal}
\end{align*}
gesetzt.

Nach Gleichung (3) ergibt sich für die Poissonsche Querkontraktionszahl
\begin{align*}
\mu = {} & \frac{E}{2G} - 1 \\
    = {} & 
\end{align*}
