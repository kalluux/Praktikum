\section{Diskussion}
\label{sec:Diskussion}

Beim Vergleich mit Literaturwerten weist der Schubmodul $G$ eine relative Abweichung von 77,81\% und
der Kompressionsmodul $Q$ eine relative Abweichung von 10\% auf, wobei der Kompressionsmodul teilweise
aus Literaturwerten berechnet wurde (Elastizitätsmodul $E$). Die Poissonsche Querkontraktionszahl $\mu$
für Eisen liegt zwischen 0,21 und 0,259, sodass diese je nach Material relativ genau bestimmt
werden konnte. Ein Vergleich des magnetischen Moments
mit Literaturwerten ist nicht möglich, wobei es jedoch in einem realistischen Bereich liegt.
Im Allgemeinen war der Versuchsaufbau sehr empfindlich, da bereits kleinste Bewegungen oder Stöße
am Tisch oder Aufbau zu verhältnismäßig großen Messabweichungen führen. Dies war gerade bei der
Messung mit eingeschaltetem Magnetfeld der Fall, da hier besonders darauf geachtet werden sollte,
dass der Draht nur um einen kleinen Winkel tordiert. Eine reine Drehschwingung ohne Pendelbewegung
zu erhalten ist mit manueller Auslenkung am Justierrad kaum möglich, wobei diese durch die Dämpfungsvorrichtung an 
der Apparatur zumindest etwas vermindert werden konnte. Systematische Messfehler durch die Photodiode oder
der automatischen Stoppuhr sind nicht zu berücksichtigen.