\section{Diskussion}
\label{sec:Diskussion}


Die Zeitkonstanten, welche mit dem Entladevorgang ermittelt wurde und mit der Phasenverschiebung, ähneln sich sehr. 
Die Anpassungsfunktionen liegen nah an den Messwerten, weshalb diese Methode nicht allzu fehleranfällig war, bis auf systematische Fehler,
wie ungenaues Ablesen der Werte auf dem Oszilloskop. Der mit der frequenzabhängigen Amplitude ermittelte Wert ist negativ, was physikalisch unlogisch ist,
und weicht außerdem stark von den anderen Zeitkonstanten ab. Hier ist also von einem großen systematischen Fehler auszugehen.
Die Messwerte liegen relativ weit ober- bzw. unterhalb der Ausgleichsgeraden, was dem oben Gesagten zustimmt. 
Desweiteren decken sich Polarplot und Messwerte nicht. Auch bei diesem Graphen wurde die Amplitude aufgetragen. Somit unterliegt 
die Messung der Amplituden starken Fehlern.

\noindent Die Integrierten der drei angelegten Spannungen, Dreiecks-, Rechtecks-, und Sinusspannung, entsprechen den erwarteten Spannungsformen.
Somit dient der RC-Kreis als Integrator.