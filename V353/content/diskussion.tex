\section{Diskussion}
\label{sec:Diskussion}


Die Zeitkonstanten, welche mit dem Entladevorgang ermittelt wurde und mit der Phasenverschiebung, also $RC_1 = (1.21 \pm 0.03)\cdot 10^{-3} \si{\second}$ 
und $RC_2 = (1.346 \pm 0.498) \cdot 10^{-3} \si{\second}$ , haben eine prozentuale Abweichung von ca. 11 Prozent. 
Die Anpassungsfunktionen liegen nah an den Messwerten, weshalb diese Methoden nicht allzu fehleranfällig waren, bis auf systematische Fehler,
wie ungenaues Ablesen der Werte auf dem Oszilloskop. Der mit der frequenzabhängigen Amplitude ermittelte Wert, $RC_3 = (0.9256\pm  0.0785)\cdot 10^{-3} \si{\second}$,
weicht um ca. 24 Prozent von $RC_1$ und um ca. 31 Prozent von $RC_2$ ab. Hier ist also von einem größeren systematischen Fehler auszugehen.
Die Messwerte liegen relativ weit ober- bzw. unterhalb der Ausgleichsgeraden, was dem oben Gesagten zustimmt. 
Desweiteren decken sich Polarplot und Messwerte nicht. Auch bei diesem Graphen wurde die Amplitude aufgetragen. Somit unterliegt 
die Messung der frequenzabhängigen Amplitude starken Fehlern, da die Zeitkonstante, welche mit jener berechnet wurde, mehr von den
sonstigen berechneten Zeitkonstanten abweicht.

\noindent Die Integrierten der drei angelegten Spannungen, Dreiecks-, Rechtecks-, und Sinusspannung, entsprechen den erwarteten Spannungsformen.
Somit dient der RC-Kreis als Integrator.