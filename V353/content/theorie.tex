\section{Theorie}
\label{sec:Theorie}

Als Relaxationszeit bezeichnet man die Zeit, die zwischen dem Auslenken eines Systems aus 
seinem Ausgangszustand und dem nicht-oszillatorischem Zurückkehren in diesen vergeht.
In diesem Versuch wird das Entladen eines Kondensators über einen Widerstand als 
Beispiel für eine Relaxationserscheinung betrachtet.
Allgemein ist die Änderungsgeschwindigkeit zur Zeit $t$ einer physikalischen Größe $A$
\begin{align}
\frac{dA}{dt} = c[A(t)-A(\infty)]
\end{align}
proportional zur Auslenkung der Größe $A$ vom (nicht erreichbaren) Endzustand $A(\infty)$.
Hieraus folgt
\begin{align}
A(t) = A(\infty)+[A(0) - A(\infty)]e^{ct},
\end{align}
wobei $c < 0$ sein muss, damit A beschränkt bleibt.

\subsection{Entladevorgang eines Plattenkondensators}
Befindet sich auf den Platten eines Kondensators mit der Kapazität $C$ eine Ladung $Q$, 
so liegt zwischen diesen die Spannung
\begin{align*}
U_\text{C} = \frac{Q}{C}.
\end{align*}
Diese bedingt nach dem Ohm'schen Gesetz einen Strom
\begin{align*}
I = \frac{U_\text{C}}{R}
\end{align*}
durch den Widerstand R um die Ladungen auszugleichen, wodurch sich die Ladung $Q$ 
pro Zeiteinheit $t$ um
\begin{align*}
dQ = -Idt
\end{align*}
verändert. Hieraus ergibt sich die zur Gleichung (1) ähnliche Gleichung
\begin{align}
\frac{dQ}{dt} = -\frac{Q(t)}{RC},
\end{align}
sodass auch für diese durch Integration und Beachtung des unerreichbaren Grenzwertes $Q(\infty) = 0$
\begin{align}
Q(t) = Q(0)e^{\frac{1}{RC}}
\end{align}
folgt. Der Ausdruck $RC$ wird als Zeitkonstante des Relaxationsvorgangs bezeichnet und 
ist maßgebend für die Geschwindigkeit, mit der dieser gegen den Endzustand $Q(\infty)$ strebt.

\subsection{Relaxationsverhalten bei angelegter Wechselspannung}
Im Allgemeinen lässt sich eine von der Kreisfrequenz $\omega$ abhängige Wechselspannung durch
\begin{align}
U(t) = U_\text{0} \cdot cos(\omega t)
\end{align}
beschreiben.
Mit zunehmender Kreisfrequenz $\omega$ bildet sich eine Phasenverschiebung $\phi$ zwischen der 
eingehenden Wechselspannung $U_\text{0}$ und der verzögerten Kondensatorspannung $U_\text{C}$
aus, sodass diese sich als
\begin{align}
U_\text{C}(t) = A(\omega) \cdot cos(\omega t + \phi(\omega))
\end{align}
ausdrücken lässt, wobei $A$ die Amplitude der Kondensatorspannung ist.
Die Stromstärke $I(t)$ lässt sich über
\begin{align}
I(t) = \frac{dQ}{dt} = C \frac{dU_\text{C}}{dt}
\end{align}
in Abhängigkeit von der Kondensatorspannung $U_\text{C}$ ausdrücken, sodass aus 
Gleichung (3) und (7) sowie den Kirchhoff'schen Regeln für die Amplitude
\begin{align}
A(\omega) = \frac{U_\text{0}}{\sqrt{1+\omega^2R^2C^2}}
\end{align}
folgt.

\subsection{Der RC-Kreis als Integrator}
Ein RC-Kreis kann unter bestimmten Umständen dazu dienen, eine zeitlich veränderliche Spannung 
$U(t)$ zu integrieren. Damit dies möglich ist, muss die Kreisfrequenz $\omega \ll \frac{1}{RC}$ sein.
Aus der Gleichung für die Gesamtspannung
\begin{align*}
U(t) = {} & U_\text{R}(t) + U_\text{C}(t) = R \cdot I(t) + U_\text{C}(t) \\
     = {} & RC \cdot \frac{dU_\text{C}(t)}{dt} + U_\text{C}(t)
\end{align*}
folgt dann
\begin{align}
U_\text{C}(t) = \frac{1}{RC} \int_0^t U(t')dt'.
\end{align}
