\section{Auswertung}
\label{sec:Auswertung}

Um die Wellenlänge des verwendeten Lasers mit Gleichung () zu berechnen, werden die Anzahl der Intensitätsmaxima $z$ und die zugehörige Verschiebung $d$ gemessen.
Die tatsächlich abgelesene Verschiebung muss noch durch die Hebelübersetzung Ü = 5,017 geteilt werden. 
Alle benötigten Daten und die daraus berechnete Wellenlänge sind in Tabelle (1) zu finden.
\begin{table}[H]
\centering
\caption{Verschiebung, Anzahl der Maxima und Wellenlänge.}
\label{tab:einzel1}
\begin{tabular}{c c c}
\toprule
$d/\si{\milli\meter}$ & $z$ & $\lambda/\si{\nano\meter}$\\
\midrule
4,3 & 3034 & 564,99 \\
4,6 & 3178 & 577,02 \\
4,8 & 3024 & 632,77 \\
4,8 & 3047 & 627,99 \\
4,9 & 3045 & 641,50 \\
4,8 & 3061 & 625,12 \\
4,8 & 3056 & 626,14 \\
4,7 & 3024 & 619,59 \\
4,8 & 3063 & 624,71 \\
4,7 & 3036 & 617,14 \\
\bottomrule
\end{tabular}
\end{table}

Der Mittelwert der Wellenlängen kann berechnet werden durch
\begin{align*}
x &= \frac{1}{10}\cdot \sum_{n=1}^{10} x_i\\
\Delta x &=\frac{1}{\sqrt{10}} \cdot \sqrt{\frac{1}{9} \sum_{n=1}^{10} (x_i - x)^2} 
\end{align*}
Die ersten beiden Werte werden dafür nicht beachtet, da sie von den übrigen deutlich abweichen. Somit beträgt der Mittelwert 
\begin{align*}
\lambda = (626,87 \pm 7,12) \si{\nano\meter} .
\end{align*}

Um den Brechungsindex von Luft zu berechen wird Formel () genutzt. Weitere Größen die bekannt sein müssen sind
\begin{align*}
T_0 &= 273.15 \,\si{\kelvin}\\
p_0 &= 1.0132 \,\si{\bar}\\
b &= 0.05 \,\si{\meter}\\
\end{align*}
wobei $T_0$ und $p_0$ die Temperatur und der Druck bei Normalbedingungen sind und $b$ die Schichtdicke.
Als Zimmertemperatur werden $295,15 \,\si{\kelvin}$ angenommen.
In Tabelle (2) sind die bei einem Druckunterschied $p-p'= 0,6 \,\si{\bar}$ gemessenen Maxima und die daraus resultierenden Brechungsindizes aufgelistet.
\begin{table}[H]
\centering
\caption{Anzahl der Maxima und berechneter Brechungsindex von Luft.}
\label{tab:einzel1}
\begin{tabular}{c c}
\toprule
$z$ & $n$\\
\midrule
24 & $1,0002696 \pm 0,0000102$ \\
23 & $1,0002584 \pm 0,0000098$ \\
24 & $1,0002696 \pm 0,0000102$ \\
24 & $1,0002696 \pm 0,0000102$ \\
25 & $1,0002809 \pm 0,0000107$ \\
25 & $1,0002809 \pm 0,0000107$ \\
25 & $1,0002809 \pm 0,0000107$ \\
24 & $1,0002696 \pm 0,0000102$ \\
24 & $1,0002696 \pm 0,0000102$ \\
25 & $1,0002809 \pm 0,0000107$ \\
\bottomrule
\end{tabular}
\end{table}
Der Mittelwert der berechneten Brechungsindizes ist
\begin{align*}
n = 1,000273 \pm 0,000010 .
\end{align*}

