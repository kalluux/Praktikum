\section{Diskussion}
\label{sec:Diskussion}

Die relative Abweichung der bestimmten Wellenlänge $\lambda =  (626,87 \pm 7,12)\,\si{\nano\meter}$ vom Theoriewert $\lambda = 635\,\si{\nano\meter}$ beträgt ca. $1,3 \%$.
Da dies eine sehr geringe Abweichung ist, kann die Messung als genau eingestuft werden. Die beiden Messwerte, welche nicht zur Berechnung verwendet wurden, unterlagen eventuellen Ablesefehlern oder auch einem ungenauen Signal am Detektor, da das ganze System sehr empfindlich war.

\noindent Der Brechungsindex von Luft beträgt ca. $n = 1,000272$ (\cite{sample1}) und der berechnete Wert $n = 1,000273$ weicht in den Nachkommastellen um ca $0,37 \%$ davon ab. Da dieser Fehler sehr gering ist, unterlag die Messung so gut wie keinen Fehlern.