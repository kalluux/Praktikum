\section{Diskussion}
\label{sec:Diskussion}

Die im ersten Teil berechneten Absorptionskoeffizienten werden im Folgenden mit den theoretisch berechneten Werten verglichen.
Dies dient zu einer Einschätzung der Messgenauigkeit, bzw. zu einer Einschätzung bezüglich der Bestätigung der Theorie.
Die experimentellen Werte zusammen mit den Theoretischen und der relativen Abweichung:
\begin{align*}
\mu_\text{Fe} = (\num{51.9 +- 1.4})\,\frac{1}{\si{\meter}} \\
\mu_{com, \text{Fe}} = 56,64\,\frac{1}{\si{\meter}} \\
\Rightarrow \text{Relative Abweichung} &= 0,7\,\% \\
\mu_\text{Fe} &= (\num{109.5 +- 2.6})\,\frac{1}{\si{\meter}} \\
\mu_{com, \text{Pb}} &= 69,36\,\frac{1}{\si{\meter}} \\
\Rightarrow \text{Relative Abweichung} &= 57,9\,\%
\end{align*}
Aufgrund der kleinen Abweichung des Absorptionskoeffizienten von Eisen, lässt sich sagen, dass bei diesem Element hauptsächlich
der Compton-Effekt bei der Absorption von $\gamma$-Strahlung eine Rolle spielt. Anders sieht es beim Element Blei aus. Wegen der hohen
Ordnungszahl und der damit verbundenen höheren Bindungsenergie der $K$-Elektronen, tritt der Compton-Effekt unwahrscheinlicher auf. Stattdessen
setzt bei steigender Ordnungszahl der Photo-Effekt viel wahrscheinlicher ein. Die Messung bestätigt somit die Theorie.
Die im zweiten Auswertungsteil ermittelte maximale Energie der Beta-Strahlung lässt sich ebenfalls mit einem Literaturwert \cite[S.20]{kent5} vergleichen:
\begin{align*}
E_\text{max, exp} &= (\num{0.30 +- 0.06})\,\si{\mega\electronvolt} \\
E_\text{max, exp} &= 0,300\,\si{\mega\electronvolt} \\
\Rightarrow \text{Relative Abweichung} &= 0\,\%
\end{align*}
In Anbetracht der bis auf Messfehler nicht vorhandenen Abweichung lässt sich festhalten, dass auch dieser Versuchsteil erfolgreich durchgeführt wurde.
Im Graph zur Bestimmung der maximalen Reichweite fehlen dadurch, dass die Taktrate
geringer war als die Nullmessung einige Stützstellen der Ausgleichsrechnung, weshalb diese
stärker von Fehlern belastet sein könnte.
Im Allgemeinen können alle Messwerte aufgrund der älteren Gerätschaften, die man 
teilweise manuell gleichzeitig zurücksetzen und starten musste, fehlerbehaftet sein.
Als letzte Fehlerquelle ergibt sich, dass bei dem Auswechseln der Platten
die Ausrichtung des Messgeräts und der Quelle verändert werden konnte, was gerade
bei dem Versuchsteil mit der Gamma-Strahlung große Auswirkungen haben kann.