\section{Diskussion}
\label{sec:Diskussion}

Im Allgemeinen fällt auf, dass die theoretisch errechneten 
Werte meist höher sind als die experimentell bestimmten. Es folgt die Bestimmung der relativen Abweichung:
\begin{align*}
R_\text{eff} &= (\num{41.0 +- 2.8})\,\si{\ohm} & R_1 &= (67.2 \pm 0.2 )\,\si{\ohm} \\
\implies \text{Relative Abweichung} &= 38,99\% \\
\end{align*}
\begin{align*}
R_\text{ap,ex} &= 3,05\,\si{\kilo\ohm} & R_\text{ap,th} &= (5.7 \pm 0.02)\,\si{\kilo\ohm}\\
\implies \text{Relative Abweichung} &= 46,49\% \\
\end{align*}
\begin{align*}
q_\text{ex} &= \num{3.78} & q_\text{th} &= (4.179 \pm 0.014)\\
\implies \text{Relative Abweichung} &= 9,55\% \\
\end{align*}
\begin{align*}
b_\text{ex} &= (6\,500 \pm 10\,000)\,\si{\hertz} & b_\text{th} &= (6\,470 \pm 40)\si{\hertz} \\
\implies \text{Relative Abweichung} &= 0,46\% \\
\end{align*}
\begin{align*}
f_\text{res,ex} &= (31,5 \pm 1)\, \si{\kilo\hertz} & f_\text{res,th} &= (26,64 \pm 0,08)\,\si{\kilo\hertz}\\
\implies \text{Relative Abweichung} &= 18,24\% \\
\end{align*}
\begin{align*}
f_\text{1,ex} &= (23 \pm 1)\,\si{\kilo\hertz} & f_\text{1,th} &= (23,99 \pm 0,07)\,\si{\kilo\hertz}\\
\implies \text{Relative Abweichung} &= 4,13\% \\
\end{align*}
\begin{align*}
f_\text{2,ex} &= (30 \pm 1)\,\si{\kilo\hertz} & f_\text{2,th} &= (30,46 \pm 0,1)\,\si{\kilo\hertz} \\
\implies \text{Relative Abweichung} &= 1,51\%.
\end{align*}
Die Abweichungen lassen sich unter anderem dadurch erklären, dass die Innenwiderstände 
sämtlicher Geräte (Spulen, Drähte, Oszilloskop) nicht miteinbezogen werden konnten, 
sodass diese Fehlerquelle in der gesamten Messung nicht auszuschließen ist. Weiterhin kann
es zu systematischen Ablesefehlern kommen, da auch die Cursorfunktion manuell bedient wird
und somit nicht exakt ist.

\noindent Im Angesicht dieser Tatsachen liegen die meisten bestimmten Werte dennoch in einem angemessenen Bereich, was sich in den 
Verläufen der Graphen widerspiegelt.