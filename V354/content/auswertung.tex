\section{Auswertung}
\label{sec:Auswertung}

\subsection{Zeitabhängigkeit der Amplitude}

%Aufgabe a
Die Form der Einhüllenden ist, da $\mathit{U(t) \propto I(t)}$, durch
\begin{align*}
A = A_\text{0} \mathrm{e}^{-2 \pi \mu t} + b
\end{align*}
gegeben.


Die aufgenommenen Wertepaare der Spannungsamplitude $U_\text{C}$ und der Zeit $t$
befinden sich in Tabelle 1. Mithilfe exponentieller Regression mittels Python erhält man
\begin{align*}
A_\text{0} &= (\num{31,32 +- 0,63})\,\mathrm{V} \\
\mu &= (\num{194.51 +- 13.15})\,\mathrm{\frac{1}{s}} .
\end{align*}

\begin{table}[H]
\centering
\caption{Messdaten zur Bestimmung des effektiven Dämpfungswiderstandes sowie der Abklingdauer.}
\label{tab:some_data}
\begin{tabular}{c c c c}
\toprule
$U_\text{C}$ \:/\: V & $t \:/\: \si{\micro\second}$ & $U_\text{C}$ \:/\: V & $t \:/\: \si{\micro\second}$\\
\midrule
20  & 32,2 & 286 & 21,2\\
58  & 29,8 & 322 & 20,6\\
96  & 27,6 & 362 & 20,4\\
134 & 25,8 & 398 & 19,8\\
172 & 24,6 & 436 & 19,4\\
210 & 23,4 & 474 & 18,8\\
248 & 22,0 \\
\bottomrule
\end{tabular}
\end{table}

Über Gleichung (10) und (13) ergibt sich dann für den effektiven Dämpfungswiderstand
$R_\text{eff}$ und die Abklingdauer $T_\text{ex}$ 
\begin{align*}
R_\text{eff} &= 4 \pi L \mu\\
       &= (\num{41.0 +- 2.8})\,\si{\ohm}\\
T_\text{ex} &= (\num{820 +- 60})\,\mathrm{µs}.
\end{align*}
Die zugehörigen Fehler ergeben sich durch die Gauß'sche Fehlerfortpflanzung der Form
\begin{align*}
\sigma_{R_\text{eff}} &= \sqrt{ 16 L^{2} \pi^{2} \sigma_{\mu}^{2} + 16 \mu^{2} \pi^{2} \sigma_{L}^{2}} \\
\sigma_{T_\text{ex}} &= \sqrt{\frac{4 L^{2} \sigma_{R_\text{eff}}^{2}}{R_\text{eff}^{4}} + \frac{4 \sigma_{L}^{2}}{R_\text{eff}^{2}}}.
\end{align*}

\begin{figure}[H]
  \centering
  \includegraphics{plota.pdf}
  \caption{Abklingvorgang des gedämpften $\mathit{RLC}$-Schwingkreises und Ausgleichsfunktion.}
  \label{fig:Plot a}
\end{figure}

\subsection{Bestimmung des Dämpfungswiderstands }
%Aufgabe b
Der Dämpfungswiderstand, für den der aperiodische Grenzfall eintrifft, wird als 
\begin{align*}
R_\text{ap,ex} = \SI{3,05}{\kilo\ohm}
\end{align*}
gemessen.
Der theoretische Wert des Dämpfungswiderstandes wird mit Gleichung (15) berechnet.
Der Fehler errechnet sich über die Gauß'sche Fehlerfortpflanzung:
\begin{align*}
\sigma_{R_\text{ap,th}} = \sqrt{\frac{\sigma_{L}^{2}\frac{L}{C}}{L^{2}} + \frac{\sigma_{C}^{2} \frac{L}{C}}{C^{2}}}
\end{align*}
Somit ergibt sich für den theoretischen Dämpfungswiderstand
\begin{align*}
R_\text{ap,th} = \SI{5.7 \pm 0.02}{\kilo\ohm}.
\end{align*}

%Aufgabe c
\subsection{Frequenzabhängigkeit der Kondensatorspannung}
Die zur Berechnung der Resonanzüberhöhung $q$ aufgenommenen Wertepaare, bestehend aus der Spannungsamplitude $U_c$ und der Frequenz $f$, befinden sich zusammen mit der
normierten Spannungsamplitude in Tabelle 2. Ferner werden die normierten Kondensatorspannungen in Abbildung 2 gegen die Frequenz aufgetragen.
Die Resonanzüberhöhung 
\begin{align*}
q_\text{ex} &= 3,78
\end{align*}
wird Tabelle 2 entnommen und ist der maximale gemessene Wert, den $\frac{U_\text{C}}{U}$ annimmt. Mit der Gleichung (22) ergibt sich für den theoretischen Wert der Resonanzüberhöhung 
\begin{align*}
q_\text{th} &= (4.179 \pm 0.014),
\end{align*}
wobei der Fehler mit der Gauß'schen Fehlerfortpflanzung in der Form
\begin{align*}
\sigma_{q_\text{th}} = \sqrt{\frac{L \sigma_{C}^{2}}{4C^{3} R^{2}} + \frac{\sigma_{L}^{2}}{4C L R^{2}} + \frac{L \sigma_{R}^{2}}{C R^{4}}}
\end{align*}
ermittelt wird.

\begin{table}
\centering
\caption{Messdaten zur Bestimmung Resonanzüberhöhung $q_\text{ex}$.}
\label{tab:some_data}
\begin{tabular}{c c c c c c c c c}
\toprule
$\mathit{U_\text{c} \: / \: \mathrm{V}}$ & $\frac{U_\text{c}}{U_0}$ & $\mathit{f \: / \: \mathrm{kHz}}$ & $\mathit{U_\text{c} \: / \: \mathrm{V}}$ & $\frac{U_\text{c}}{U_0}$ & $\mathit{f \: / \: \mathrm{kHz}}$ & $\mathit{U_\text{c} \: / \: \mathrm{V}}$ & $\frac{U_\text{c}}{U_0}$ & $\mathit{f \: / \: \mathrm{kHz}}$ \\
\midrule
2  & 1     & 10,2 & 23 & 2,980  & 30,4 & 36 & 1,019  & 10,6 \\
4  & 1,019  & 10,4 & 24 & 3,333  & 34 & 38 & 0,846  &  8,8 \\
6  & 1,038  & 10,8 & 25 & 3,7    & 37 & 40 & 0,711  &  7,4 \\
8  & 1,056  & 11,2 & 26 & 3,78   & 37,8 & 42 & 0,596  &  6,2 \\
10 & 1,074  & 11,6 & 27 & 3,58   & 35,8 & 44 & 0,519  &  5,4 \\
12 & 1,169  & 12,4 & 28 & 3,18   & 31,8 & 46 & 0,452  &  4,8 \\
14 & 1,283  & 13,6 & 29 & 2,705  & 27,6 & 48 & 0,396  &  4,2 \\
16 & 1,433  & 15,2 & 30 & 2,313  & 23,6 & 50 & 0,358  &  3,8 \\
18 & 1,660  & 17,6 & 31 & 1,942  & 20,2 & 52 & 0,320  &  3,4 \\
20 & 2,019  & 21 & 32 & 1,673  & 17,4 & & & \\
22 & 2,423  & 25,2 & 34 & 1,288  & 13,4 & & & \\
\bottomrule
\end{tabular}
\end{table}

\begin{figure}[H]
  \centering
  \includegraphics{plotc.pdf}
  \caption{Lineare Darstellung der normierten Kondensatorspannung $\frac{U_\text{C}}{U_\text{0}}$ in Abhängigkeit der Frequenz $f$.}
  \label{fig:Plot c}
\end{figure}

Der experimentelle Wert der Halbwertsbreite $b_\text{ex}$ wird aus der Abbildung als
\begin{align*}
b_\text{ex} = (6\,500 \pm 100)\,\si{\hertz}
\end{align*}
näherungsweise abgelesen, da auf der Höhe $\frac{U_\text{max}}{\sqrt{2}}$ keine zugehörigen Frequenzen gemessen wurden.
Hierzu wurde die Länge des Balkens $b$ in 100 Hz Schritten an die Kurve angeglichen. Mit Gleichung (24) wird die theoretische Halbwertsbreite 
\begin{align*}
b_\text{th} &= (6\,470 \pm 40)\,\si{\hertz}
\end{align*}
bestimmt. Der zugehörige Fehler wird über die Gauß'sche Fehlerfortpflanzung 
\begin{align*}
\sigma_{b_\text{th}} = \sqrt{\frac{\sigma_{R}^{2}}{ L^{2}} + \frac{R^{2} \sigma_{L}^{2}}{L^{4}}}
\end{align*}
ermittelt.


%Aufgabe d
\subsection{Frequenzabhängigkeit der Phasenverschiebung}
In Abbildung 9 erkennt man die für die Phasenverschiebung benötigten Parameter a und b. Die Phasenverschiebung lässt sich durch
\begin{align*}
\phi = \frac{a}{b} \cdot 2\pi
\end{align*}
berechnen und wird zusammen mit den Messwerten für die Frequenzabhängigkeit der Phasenverschiebung in Tabelle 3 festgehalten.
Ferner wird in Abbildung 12 die Phasenverschiebung $\phi$ halblogarithmisch gegen die Frequenz $f$ aufgetragen.


Mittels Gleichung (21) wird die theoretische Resonanzfrequenz $f_\text{res,th}$ bestimmt. 
Die Frequenzen $f_\text{1,th}$ und $f_\text{2,th}$ für die Phasenverschiebung $\frac{\pi}{4}$
und $\frac{3\pi}{4}$ werden nach Gleichung (19) berechnet. Die entsprechenden experimentell
ermittelten Werte werden der Tabelle 3 entnommen, indem die Frequenz notiert wird,
die am nähesten an dem erwünschten Wert, also dem Maximum oder der bestimmten Phasenverschiebung liegt.
 Es ergibt sich:
\begin{align*}
f_\text{res,ex} &= (31,5 \pm 1)\, \si{\kilo\hertz},\\
f_\text{res,th} &= (26,64 \pm 0,08)\,\si{\kilo\hertz},\\
f_\text{1,ex} &= (23 \pm 1)\,\si{\kilo\hertz},\\
f_\text{1,th} &= (23,99 \pm 0,07)\,\si{\kilo\hertz},\\
f_\text{2,ex} &= (30 \pm 1)\,\si{\kilo\hertz},\\
f_\text{2,th} &= (30,46 \pm 0,1)\,\si{\kilo\hertz}.
\end{align*}

\begin{table}
\centering
\caption{Messdaten zur Bestimmung der frequenzabhängigen Phasenverschiebung.}
\label{tab:some_data}
\begin{tabular}{c c c c c c c c c}
\toprule
$f\:/\:\si{\kilo\hertz}$ & $a\:/\:\si{\micro\second}$ & $\phi\:/\:\text{rad}$ & $f\:/\:\si{\kilo\hertz}$ & $a\:/\:\si{\micro\second}$ & $\phi\:/\:\text{rad}$ & $f\:/\:\si{\kilo\hertz}$ & $a\:/\:\si{\micro\second}$ & $\phi\:/\:\text{rad}$ \\
\midrule
2  & 1,64 & 0,0206  & 23 & 5,20 & 0,7515 & 36 & 12,3 & 2,7822 \\
4  & 1,48  & 0,0372 & 24 & 6,32 & 0,9530 & 38 & 12 &  2,8651 \\
6  & 1,68  & 0,0633 & 25 & 7,60 & 1,1939 & 40 & 11,5 &  2,8903 \\
8  & 1,53  & 0,0769 & 26 & 9,20 & 1,5029 & 42 & 11,1  &  2,9292 \\
10 & 1,68  & 0,1056 & 27 & 10,60 & 1,7982 & 44 & 10,7  &  2,9581 \\
12 & 1,84  & 0,1387 & 28 & 11,7 & 2,0584 & 46 & 10,3  &  2,9770 \\
14 & 1,91  & 0,1680 & 29 & 12,3 & 2,2412 & 48 & 10  &  3,0160 \\
16 & 2,06  & 0,2071 & 30 & 12,8 & 2,4127 & 50 & 9,5  &  2,9845 \\
18 & 2,28  & 0,2579 & 31 & 12,9 & 2,5126 & 52 & 9,4  &  3,0712 \\
20 & 2,66  & 0,3343 & 32 & 12,9 & 2,5937 & & & \\
22 & 3,40  & 0,4700 & 34 & 12,7 & 2,7131 & & & \\
\bottomrule
\end{tabular}
\end{table}

\begin{figure}
  \centering
  \includegraphics{plotd.pdf}
  \caption{Phasenverschiebung $\phi$ in Abhängigkeit der Frequenz $f$.}
  \label{fig:Plot d}
\end{figure}