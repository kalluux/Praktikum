\section{Auswertung}
\label{sec:Auswertung}

\subsection{Überprüfung der Bragg Bedingung}
Die Messreihe ist in Abbildung (3) dargestellt.
Misst man das Maximum aus, ergibt sich ein doppelter Winkel von $2 \cdot \theta = 27,5$°, also ein Bragg-Winkel von $\theta = 13,75$° Da für den Kristall ein fester Winkel von $14$° eingestellt wurde, müsste der Winkel bei $\theta = 14$° liegen.


\subsection{Analyse des Emissionsspektrums einer Kupfer-Röntgenröhre}
Die aufgetragene Intensität in Abhängigkeit des doppelten Winkels $2 \cdot \theta$ ist in Abbildung (4) zu sehen.
Der von links gesehen erste Peak entspricht der $K_{\beta}$-Linie, der zweite der $K_{\alpha}$-Linie.
Aus dem Grenzwinkel $\theta = 4$° wird die maximale Energie des Bremsspektrums mit Gleichung (1) und (5) bestimmt

\begin{align*}
E_{max} = \frac{hc}{2\cdot d \sin{\theta}} = 44,129 \si{\keV} .
\end{align*}

\noindent Um die Halbwertsbreite zu bestimmen werden die Winkel auf der Mitte der Peaks gemessen. 
Diese betragen bei der $K_{\beta}$-Linie
\begin{align*}
\theta_{\beta,1} = 19,845° \\
\theta_{\beta,2} = 20,315°
\end{align*}
und bei der $K_{\alpha}$-Linie
\begin{align*}
\theta_{\alpha,1} = 22,03° \\
\theta_{\alpha,2} = 22,50° .
\end{align*}
Also lauten die Halbwertsbreiten
\begin{align*}
b_{\alpha} &= \frac{\theta_{\alpha,2} - \theta_{\alpha,1}}{2} = 0,235° \\
b_{\beta} &= 0,235° .
\end{align*}

\noindent Das Auflösungsvermögen ergibt sich durch die Energiedifferenzen der gemessenen Winkel bei der Halbwertsbreite. Die Energien werden wie oben bestimmt.
\begin{align*}
\Delta E_{\theta_{\alpha}} = E_{\theta_{\alpha,1}} - E_{\theta_{\alpha,2}} = (8,207 - 8,044) \si{\keV} = 0,163 \si{\keV}. \\
\Delta E_{\theta_{\beta}} = E_{\theta_{\beta,1}} - E_{\theta_{\beta,2}} = (9,068 - 8,866) \si{\keV} = 0,201 \si{\keV}. 
\end{align*}


\noindent Um die Abschirmkonstanten zu bestimmen, muss zuerst die Energie $E_\alpha$ und $E_\beta$ berechnet werden. 
Die Maxima der beiden Peaks liegen bei den Winkeln $\theta_{\beta} = $° und $\theta_{\alpha} = $°.
Daraus ergeben sich erneut mit Gleichungen (1) und (5) folgende Energien
\begin{align*}
E_\alpha &=  \\
E_\beta &= .
\end{align*}
Mit Gleichungen (2) und (3) werden die Abschirmkonstanten für die $K_{\alpha}$-Linie und die $K_{\beta}$-Linie berechnet.
\begin{align*}
\sigma_1 &= z_{Cu} - \sqrt{\frac{E_\beta}{R_\infty}} = \\
\sigma_2 &= z_{Cu} - 2 \cdot \sqrt{\frac{R_\infty \cdot (z_Cu - \sigma_1)^2 - E_\alpha}{R_\infty}} = \\
\sigma_3 &= z_{Cu} - 3 \cdot \sqrt{\frac{R_\infty \cdot (z_Cu - \sigma_1)^2 - E_\beta}{R_\infty}} = 
\end{align*}
Dabei entspricht $\sigma_2$ der Abschirmkonstanten auf der $K_{\alpha}$-Linie und $\sigma_3$ der auf der $K_{\beta}$-Linie.
