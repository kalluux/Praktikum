\section{Diskussion}
\label{sec:Diskussion}

Die gemessene 1. Anregungsenergie und der Literaturwert (\cite{kent3})
\begin{align*}
E_1 = (4,89 \pm 0,206) \si{\eV} \\
E_{1lit} = 4,9 \si{\eV}
\end{align*}
sind fast identisch. Die Abweichung beträgt nur ca. $0,2$ Prozent.
Somit lässt sich sagen, dass die Werte für die Abstände gut abgemessen werden konnten. 
Die Maxima waren besonders gut für hohe Temperaturen zu erkennen.

Die Ionisierungsspannung symbolisiert im Graphen einen signifikanten Anstieg der Stromstärke.
Verglichen mit dem Literaturwert (\cite{kent2}) liegen sie bei
\begin{align*}
U_{ion} = 0,95\si{\V} \\
U_{ion,lit} = 10,438\si{\V}.
\end{align*}
