\section{Diskussion}
\label{sec:Diskussion}

Insgesamt konnte keine Messung bei einer konstanten Temperatur durchgeführt werden, da sie immer schwankte.
Die Spannungen konnten auch nur ungenau abgelesen werden, da dies ohne wirkliche Hilfsmittel stattfinden musste.
Die gemessene 1. Anregungsenergie und der Literaturwert\cite{kent3}
\begin{align*}
E_1 = (4,89 \pm 0,206) \si{\eV} \\
E_{1lit} = 4,9 \si{\eV}
\end{align*}
sind fast identisch. Die Abweichung beträgt nur ca. $0,2$ Prozent.
Somit lässt sich sagen, dass die Werte für die Abstände gut abgemessen werden konnten. 
Die Maxima waren besonders gut für hohe Temperaturen zu erkennen.
Der berechnete Wert für die Wellenlänge $\lambda = (2,54 \pm 0,11)\cdot 10^{-7} \si{\m}$ liegt außerdem in einem sinnvollen Bereich, und es handelt sich
somit um UV-Strahlung.

Die Ionisierungsspannung symbolisiert im Graphen einen signifikanten Anstieg der Stromstärke.
Verglichen mit dem Literaturwert\cite{kent2} liegen sie bei
\begin{align*}
U_{ion} = 0,95\si{\V} \\
U_{ion,lit} = 10,438\si{\V}.
\end{align*}
Das ergibt eine Abweichung von ca $90$ Prozent. Der Fehler muss bei der Aufzeichnung mit dem XY-Schreiber aufgetreten sein. Der Anstieg der Kurve
liegt auch nur im Bereich von einer Spannung zwischen $ (2 - 4) \si{\V}$, was viel zu niedrig ist. 
