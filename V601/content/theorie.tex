\section{Zielsetzung}
\label{sec:Zielsetzung}
Bei dem Franck-Hertz-Versuch wird die Energiedifferenz
zwischen dem ersten angeregten Zustand und dem Grundzustand sowie die Ionisationsenergie eines
Hg-Atoms bestimmt. Weiter wird die Energieverteilung der beschleunigten Elektronen untersucht.


\section{Theorie}
\label{sec:Theorie}
Der Franck-Hertz-Versuch gehört zu den Elektronenstoßexperimenten, mit denen die
diskrete Struktur der Elektronenhüllen untersucht werden kann. In einer abgeschlossenen
 Kammer kommt es dabei zu Wechselwirkungen von Elektronen einer bestimmten Energie und
Hg-Atomen. Diese Wechselwirkungen sind elastische und inelastische Stöße, wobei bei
letzteren die Hg-Atome in den ersten angeregten Zustand versetzt werden. Dabei
nehmen sie die Energie auf, die sich aus der kinetischen Energiedifferenz der Elektronen vor und
nach dem Stoß ergibt:
\begin{align}
\frac{m_\text{0} v_\text{vor}^{2}}{2} - \frac{m_\text{0} v_\text{nach}^{2}}{2} = E_\text{1} - E_\text{0},
\end{align}
mit der Ruhemasse eines Elektrons $m_\text{0}$ und den Geschwindigkeiten $v_\text{vor}$
und $v_\text{nach}$ vor und nach dem Stoß. Die Energie der Elektronen wird mithilfe der
Gegenfeldmethode bestimmt.

\subsection{Versuchsaufbau und Gegenfeldmethode}
In Abbildung 1 ist eine schematische Darstellung der Franck-Hertz-Appartur zu sehen.
Das evakuierte Gefäß beinhaltet einen Tropfen Quecksilber, welches teilweise spontan verdampft
und für einen Gleichgewichtsdampfdruck $p_\text{Sättigung}$ sorgt, welcher nur von der
Umgebungstemperatur abhängig ist.
Der Glühdraht wird mittels Gleichstrom erhitzt, sodass durch den glühelektrischen Effekt
Elektronen austreten. Dem gegenüber steht eine Elektrode, an welcher eine positive
Spannung $U_\text{B}$ anliegt, sodass die Elektronen beschleunigt werden und anschließend
eine kinetische Energie
\begin{align}
\frac{v_\text{vor}^{2}}{2} = e_\text{0} U_\text{B},
\end{align}
wobei $e_\text{0}$ die Elementarladung ist, besitzen, sofern sie vorher die Geschwindigkeit $v = 0$ hatten.
Hinter diese Beschleunigungselektrode wird eine Auffängerelektrode gesetzt. 
Diese ist gegenüber der Beschleunigungselektrode negativ geladen, 
sodass die Elektronen in diesem Teil des Aufbaus abgebremst werden. Nur die Elektronen,
 deren Geschwindigkeit $v_\text{z}$ in Feldrichtung die Ungleichung
\begin{align*}
\frac{m_\text{0}}{2} v_\text{z}^{2} \leq e_\text{0} U_\text{A}
\end{align*}
erfüllt, kommen an der Auffängerelektrode an, die anderen kehren zur Beschleunigungselektrode zurück.

Im Beschleunigungsraum kommt es zwischen den Elektronen und den Hg-Atomen zu
unterschiedlichen Stößen. Ist die Energie der Elektronen gering, treten nur elastische Stöße
auf, bei denen der Energieverlust $\delta E$ des Elektrons nicht von Relevanz ist, da das
Massenverhältnis $\frac{m_\text{0}}{M}$ , welches den Energieverlust bestimmt, sehr klein ist:
\begin{align*}
\delta E = \frac{4 m_\text{0} M}{(m_\text{0} + M)^{2}} \cdot E \approx 1,1 \cdot 10^{-5} E
\end{align*}
wobei $E$ die Energie des Elektrons ist. Während der Energieverlust also zu vernachlässigen ist, 
ist die Richtungsänderung, die das Elektron bei dem Stoß erfährt, trotzdem
relevant. 
Wird die Energie der Elektronen so groß wie die Energiedifferenz
$E_\text{1} - E_\text{0}$ oder größer, kommt es zu inelastischen Stößen zwischen ihnen und den
Hg-Atomen. Auf diese wird der Betrag der Energiedifferenz übertragen, wodurch
sie angeregt werden. Die restliche Energie behält das Elektron. Das Hg-Atom geht vom
ersten angeregten Zustand unter Emission eines Lichtquants mit der Energie 
\begin{align}
h \nu = E_\text{1} - E_\text{0},
\end{align}
wobei $h$ das Plancksche Wirkungsquantum und $\nu$ die Frequenz der emittierten Strahlung
ist, wieder in den Grundzustand zurück.

