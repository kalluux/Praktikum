\section{Diskussion}
\label{sec:Diskussion}

Im Allgemeinen fällt auf, dass die theoretisch errechneten 
Werte höher sind als die experimentell bestimmten. Es folgt die Bestimmung der relativen Abweichung:
\begin{align*}
B_\text{KS, exp} &= 1,838\,\si{\milli\tesla}, & B_\text{KS, th} &= 2,055\,\si{\milli\tesla}, \\
\implies \text{Relative Abweichung} &= 10,56\%, \\
\end{align*}
\begin{align*}
B_\text{LS, exp} &= 2,374\,\si{\milli\tesla}, & B_\text{LS, th} &= 2,424\,\si{\milli\tesla}, \\
\implies \text{Relative Abweichung} &= 2,06\%, \\
\end{align*}\begin{align*}
B_\text{Hd2A, exp} &= 1,435\,\si{\milli\tesla}, & B_\text{Hd2A, th} &= 1,637\,\si{\milli\tesla}, \\
\implies \text{Relative Abweichung} &= 12,34\%, \\
\end{align*}\begin{align*}
B_\text{Hd4A, exp} &= 3,004\,\si{\milli\tesla}, & B_\text{Hd4A, th} &= 3,274\,\si{\milli\tesla}, \\
\implies \text{Relative Abweichung} &= 8,25\%, \\
\end{align*}\begin{align*}
B_\text{Hr, exp} &= 2,729\,\si{\milli\tesla}, & B_\text{Hr, th} &= 2,877\,\si{\milli\tesla}, \\
\implies \text{Relative Abweichung} &= 5,14\%. \\
\end{align*}
Die Abweichungen lassen sich unter anderem dadurch erklären, dass 
die theoretischen Werte mittels Gleichungen für lange Spulen bestimmt
wurden, in welchen die Flussdichte ab einer gewissen Tiefe homogen ist.
Dies hat gerade bei der kurzen Spule große Auswirkungen auf die Abweichung,
was dadurch bestätigt wird, dass die Abweichung bei der langen Spule deutlich
geringer ist, wobei diese wohl auch noch nicht lang genug ist, um durch die
Gleichung beschrieben zu werden.
Weiterhin konnten die Innenwiderstände
der Gerätschaften (Spulen, Drähte) nicht miteinbezogen werden, sodass
diese Fehlerquelle in der gesamten Messung nicht auszuschließen ist.
Im Angesicht dieser Tatsachen liegen die meisten bestimmten Werte dennoch in einem
angemessenen Bereich, was sich in den Verläufen der Graphen widerspiegelt.