\section{Auswertung}
\label{sec:Auswertung}

\begin{table}[H]
  \centering
  \caption{Magnetische Flussdichte und Abstand einer langen Spule.}
  \begin{tabular}{c c}
    \toprule
     $x/cm$ & $B/mT$  \\
    \midrule
    -4 & 0,131 \\
    -3 & 0,193 \\
    -2 & 0,316 \\
    -1 & 0,569 \\
    0 & 1,033\\
    1 & 1,648 \\
    2 & 2,007 \\
    3 & 2,185 \\
    4 & 2,280 \\
    5 & 2,330 \\
    6 & 2,353 \\
    7 & 2,369 \\
    8 & 2,374 \\
    9 & 2,372 \\
    10 & 2,364 \\
    11 & 2,346 \\
    12 & 2,307 \\
  \bottomrule
  \end{tabular}
\end{table}

\begin{table}[H]
  \centering
  \caption{Magnetische Flussdichte und Abstand einer kurzen Spule.}
  \begin{tabular}{c c}
    \toprule
     $x/cm$ & $B/mT$  \\
    \midrule
    -5 & 0,039 \\
    -4 & 0,079 \\
    -3 & 0,147 \\
    -2 & 0,287 \\
    -1 & 0,567 \\
    0 & 1,130\\
    1 & 1,551 \\
    2 & 1,801 \\
    3 & 1,838 \\
    4 & 1,684 \\
    5 & 1,351 \\
    6 & 0,728 \\
    7 & 0,377 \\
    8 & 0,173 \\
    9 & 0,100 \\
    10 & 0,060 \\
    11 & 0,035 \\
    12 & 0,018 \\
  \bottomrule
  \end{tabular}
\end{table}


\begin{table}[H]
  \centering
  \caption{Magnetische Flussdichte und Stromstärke einer Ringspule.}
  \begin{tabular}{c c | c c | c c | c c | c c}
    \toprule
     $I/A$ & $B/mT$ &$I/A$ & $B/mT$ & $I/A$ & $B/mT$ &$I/A$ & $B/mT$ & $I/A$ & $B/mT$  \\
    \midrule
    0 &-2,562 & 9 & 673,6 & -1 & -0,724 & -9 & -673,9 & 1 & 75,66\\
    1 & 134,2 & 8 & 655,7 & -2 & -242,8 & -8 & -656,0 & 2 & 246,5\\
    2 & 292,3 & 7 & 636,5 & -3 & -373,8 & -7 & -636,9 & 3 & 382,7\\
    3 & 401,1 & 6 & 613,7 & -4 & -463,2 & -6 & -613,8 & 4 & 466,1\\
    4 & 476,5 & 5 & 585,8 & -5 & -523,2 & -5 & -585,6 & 5 & 525,7\\
    5 & 529,8 & 4 & 551,0 & -6 & -569,5 & -4 & -551,1 & 6 & 571,6\\
    6 & 573,2 & 3 & 504,7 & -7 & -605,5 & -3 & -506,6 & 7 & 606,3\\
    7 & 608,4 & 2 & 444,3 & -8 & -636,5 & -2 & -443,6 & 8 & 636,5\\
    8 & 638,4 & 1 & 315,1 & -9 & -664,2 & -1 & -314,6 & 9 & 664,0\\
    9 & 664,8 & 0 & 122,2 & -10 & -689,4 & 0 & -122,0 & 10 & 687,5\\
    10 & 688,8 \\
   
    
  \bottomrule
  \end{tabular}
\end{table}


\begin{figure}
  \centering
  \includegraphics{spule_lang.pdf}
  \caption{Plot.}
  \label{fig:plot}
\end{figure}

\begin{figure}
  \centering
  \includegraphics{spule_kurz.pdf}
  \caption{Plot.}
  \label{fig:plot}
\end{figure}

\begin{figure}
  \centering
  \includegraphics{helmholtzD.pdf}
  \caption{Plot.}
  \label{fig:plot}
\end{figure}

\begin{figure}
  \centering
  \includegraphics{helmholtzR.pdf}
  \caption{Plot.}
  \label{fig:plot}
\end{figure}

\begin{figure}
  \centering
  \includegraphics{hysterese.pdf}
  \caption{Plot.}
  \label{fig:plot}
\end{figure}
