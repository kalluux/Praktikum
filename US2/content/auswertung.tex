\section{Auswertung}
\label{sec:Auswertung}

\subsection{Bestimmung der Größe der Fehlstellen im Acrylblock}

Die Höhe $h$ des Acrylblocks beträgt $8,35\,\si{\centi\meter}$ 
und die Schallgeschwindigkeit in Acryl beträgt $c_A=2730\,\si{\meter\per\second}$.
Die gemessene Zeit des A-Scans ohne Fehlstelle beträgt $t=60,5\,\si{\micro\second}$.

Mithilfe von Gleichung 7 ergibt sich die ermittelte Höhe des Blocks zu
\begin{align*}
h_\text{A} = \frac{1}{2} c_\text{A} t = 0,0826\,\si{\meter} = 8,26\,\si{\centi\meter}.
\end{align*}

In Tabelle 1 befinden sich die Schalllaufzeiten des A-Scans von der oberen $t_\text{A,o}$ 
und unteren $t_\text{A,u}$ Kante zur Fehlstelle sowie 
die damit und mithilfe der Gleichung 7 berechneten Abstände zur oberen $s_\text{A,o}$ und unteren
$s_\text{A,u}$ Kante, abzüglich der $0,2\,\si{\centi\meter}$ dicken Schutzschicht der Sonde, dessen
Verzögerung in jeden Scan einfließt. Aus den beiden Abständen 
lassen sich die Durchmesser der Fehlstellen 
\begin{align*}
d_\text{A} = h_\text{A} - (s_\text{A,o} + s_\text{A,u})
\end{align*}
berechnen, welche ebenfalls in Tabelle 1 aufgelistet sind.

\begin{table}[H]
  \centering
  \caption{Gemessene Abstände der Fehlstellen in dem Acrylblock.}
  \begin{tabular}{c c c c c c}
    \toprule
  Fehlstelle & $t_o/\symup{\mu}$s & $t_u/\symup{\mu}$s & $s_\text{A,o}/$cm & $s_\text{A,u}/$cm & $d_\text{A}$cm\\
    \midrule
    3  & 46,7 &  11,9  & 5,33 & 1,21  & 1,71  \\
    4  & 41,4 &  18    & 4,71 & 1,93  & 1,62  \\
    5  & 36   &  24,1  & 4,07 & 2,66  & 1,53  \\
    6  & 30,5 &  30,5  & 3,41 & 3,41  & 1,43  \\
    7  & 24,5 &  36,3  & 2,70 & 4,10  & 1,45  \\
    8  & 18,9 &  42,1  & 2,04 & 4,79  & 1,43  \\
    9  & 13,1 &  47,9  & 1,35 & 5,48  & 1,43  \\
    10 & 7,5  &  53,7  & 0,69 & 6,16  & 1,40  \\
    11 & 42,5 &  13,3  & 4,84 & 1,38  & 2,04  \\
    \bottomrule
  \end{tabular}
\end{table}

\subsection{Untersuchung des Auflösungsvermögens}
In Abbildung 3 bis 5 ist der A-Scan mit Sonden unterschiedlicher Frequenz
von der oberen Kante zu den Fehlstellen 1 und 2 dargestellt, an dessen Positionen
sich die Cursor befinden.

\begin{figure}[H]
  \centering
  \includegraphics[height=7cm]{1mhz.png}
  \caption{Untersuchung der Löcher 1 und 2 mit Sonde mit 1MHz.}
  \label{fig:1mhz}
\end{figure}

\begin{figure}[H]
  \centering
  \includegraphics[height=7cm]{2mhz.png}
  \caption{Untersuchung der Löcher 1 und 2 mit Sonde mit 2MHz.}
  \label{fig:2mhz}
\end{figure}

\begin{figure}[H]
  \centering
  \includegraphics[height=7cm]{4mhz.png}
  \caption{Untersuchung der Löcher 1 und 2 mit Sonde mit 4MHz.}
  \label{fig:4mhz}
\end{figure}


\subsection{Bestimmung der Abmessungen der Störstellen mittels B-Scan}
In Abbildung 6 und 7 ist der B-Scan des Blocks dargestellt. Mit der Eingabe der Schallgeschwindigkeit in Acryl
$c_\text{A}$ in das verwendete Programm lassen sich die Abstände zur oberen $s_\text{B,o}$ und
unteren $s_\text{B,u}$ Kante auslesen. Aus diesen lässt sich
erneut der Durchmesser der Fehlstellen $d_\text{B}$ bestimmen. Die genannten Abstände und die
Durchmesser sind in Tabelle 2 notiert.

\begin{figure}[H]
  \centering
  \includegraphics[height=7cm]{b-time-1.png}
  \caption{Abbildung des B-Scans von der oberen Kante.}
  \label{fig:acryl}
\end{figure}

\begin{figure}[H]
  \centering
  \includegraphics[height=7cm]{b-time-2.PNG}
  \caption{Abbildung des B-Scans von der unteren Kante.}
  \label{fig:acryl}
\end{figure}

\begin{table}[H]
  \centering
  \caption{Mittels B-Scan berechneten Abstände und Durchmesser.}
  \begin{tabular}{c c c c}
    \toprule
  Fehlstelle & $s_\text{B,o}/$cm & $s_\text{B,u}/$cm & $d_\text{B}/$cm\\
    \midrule
    3   & 6,24 & 1,49  & 1,02   \\
    4   & 5,5  & 2,33  & 0,92   \\
    5   & 4,77 & 3,15  & 0,83   \\
    6   & 4,03 & 4,03  & 0,69   \\
    7   & 3,22 & 4,83  & 0,7    \\
    8   & 2,42 & 5,61  & 0,72   \\
    9   & 1,64 & 6,41  & 0,7  \\
    \bottomrule
  \end{tabular}
\end{table}




\subsection{Bestimmung des Herzvolumens}
In Tabelle \ref{tab:1} werden für die 7 simulierten Herzschläge jeweils die gemessenen Amplituden $t_s$ und die daraus berechneten
Volumina $V_s$ aufgeführt. In Abbildung \ref{fig:TM} ist das Diagramm des erstellten TM-Scans zu erkennen, aus dem die besagten Amplituden
bestimmt wurden. Das Luftvolumen, welches das Wasser beim Pumpen durch die bewegliche Membran verdrängt, wird als
Kugelsegment angenähert und ergibt sich zu
\begin{equation*}
  V_s = \frac{h\pi}{6}\cdot(3r^2 + h^2),
\end{equation*}
mit
\begin{equation*}
h = \frac{1}{2}\cdot c \cdot t_\text{s}.
\end{equation*}
Der Radius beträgt $r = 2,5 \: \symup{cm}$ und die Frequenz des simulierten
Herzschlags beträgt $f = 0,45\,\si{\hertz}$.

\begin{figure}[H]
  \centering
  \includegraphics[height=7cm]{tm-herz.png}
  \caption{Abbildung des TM-Scans für den simulierten Herzschlag.}
  \label{fig:TM}
\end{figure}

\begin{table}[H]
  \centering
  \caption{Gemessene Amplituden bei Herzschlagsimulation.}
  \label{tab:1}
  \begin{tabular}{c c c }
    \toprule
  Schlag & $t_s/\symup{\mu s}$ & $V_s/\symup{cm^3}$ \\
    \midrule
    1  &  10,98 & 16,48     \\
    2  &  12,72 & 19,79     \\
    3  &  13,29 & 20,94     \\
    4  &  12,14 & 18,65     \\
    5  &  12,42 & 19,20     \\
    6  &  12,79 & 19,93     \\
    7  &  11,56 & 17,55     \\
    \bottomrule
  \end{tabular}
\end{table}

\noindent Als Mittelwert ergibt sich ein Volumen von $V_{m} = (18,93 \pm 1,41) \: \symup{cm^3}$.

\noindent Aus
\begin{equation*}
  V_{Herz} = V_m \cdot f
\end{equation*}
ergibt sich das gesuchte Herzvolumen zu
\begin{equation*}
  V_{Herz} = (8,5 \pm 0,6) \: \symup{\frac{cm^3}{s}}.
\end{equation*}

