\section{Diskussion}
\label{sec:Diskussion}

Der Theoretische Wert für die verwendete Sammellinse beträgt $f=100\si{\milli\meter}$.
Der bestimmte Wert für die Brennweite nach der ersten Methode und seine relative Abweichung zur Herstellerangabe lauten
\begin{align*}
f_1 &= (98,08 \pm 0,57) \si{\milli\meter}\\
\sigma_1 &= 1,92 \%
\end{align*}
Der graphisch abgelesene und gemittelte Wert von $f = (97,75 \pm 1,25)\si{\milli\meter}$ weicht um ca. $2,25 \%$ vom Theoriewert ab.


Die Brennweiten für weißes Licht, welche nach der Methode von Bessel bestimmt wurden 
\begin{align*}
f_{W1} = (98,70 \pm 0,53) \si{\milli\meter} \\
f_{W2} = (96,55 \pm 0,34) \si{\milli\meter} 
\end{align*}
weichen um $1,3 \%$ und um $3,45 \%$ von dem Theoriewert ab. 

Die Brennweiten für rotes Licht lauten
\begin{align*}
f_{R1} = (100,19 \pm 0,55) \si{\milli\meter} \\
f_{R2} = (97,08 \pm 0,38) \si{\milli\meter} 
\end{align*}
und weichen um $0,19 \%$ und um $2,92 \%$ vom Theoriewert ab.

Die Brennweiten für blaues Licht
\begin{align*}
f_{B1} = (98,33 \pm 0,59) \si{\milli\meter} \\
f_{B2} = (96,26 \pm 0,31) \si{\milli\meter} .
\end{align*}
weichen um $1,67 \%$ und um $3,74 \%$ von dem Theoriewert ab.

Die graphische Bestimmung der Brennweiten nach der Methode von Abbe liefern Werte von 
\begin{align*}
f = (171,39 \pm 5,71)\si{\milli\meter} \\
f = (178,19 \pm 4,92)\si{\milli\meter}.
\end{align*}
Diese weichen um $2,83 \%$ und um $6,91 \%$ von dem Theoriewert $f_T = 166,67 \si{\milli\meter}$ ab.

Insgesamt lässt sich sagen, dass die Fehler relativ gering und somit von einer guten Messung ausgegangen werden kann. Allerdings ist Hauptfehlerquelle
das ungenaue Erkennen, ob das Bild scharf gestellt ist oder nicht. Bei Verschiebung ändert sich die Schärfe des Bildes nur minimal. Außerdem wurde
die Bildgröße nur mit einem Geodreieck abgemessen, was wiederum ein systematischer Fehler ist.
Die aufgetragenen Messwerte aus der ersten Messung liefern keinen eindeutigen Schnittpunkt im Diagramm, was das oben gesagte unterstützt.

Bei der Messung mit einem Blaufilter wird erwartet, dass die Brennweite kleiner ist und mit einem Rotfilter die Messung größer ausfällt. In diesem Fall sind die mit blauem Licht gemessenen Brennweiten kleiner als die mit weißem Licht gemessenen, allerdings nur um einen sehr minimalen Wert. Gleiches lässt
sich über den Rotfilter anders herum sagen. Trotzdem liegt diese Brennweite nicht weit über dem eigentlichen Wert. Somit kann der Einfluss der unterschiedlichen Wellenlängen nicht wirklich nachgewiesen werden.

Bei den Kurven nach der Methode von Abbe sind Linearitäten zu erkennen. Allerdings weichen die Brennweiten stark vom Theoriewert ab. Untereinander sind sie auch nicht identisch, was eigentlich der Fall sein sollte. Da hier die Bildweite mit dem Geodreieck gemessen wurde und größere Abstände zwischen Bild und Gegenstand gemessen wurden, sind hier die oben genannten Fehlerquellen am stärksten.

