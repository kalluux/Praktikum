\section{Durchführung}
\label{sec:Durchführung}

In dieser Versuchsanordnung werden eine optische Bank, eine Halogenlampe,
diverse Linsen, ein Schirm, sowie ein Gegenstand ('Perl L') verwendet.
In der ersten Messung werden für zehn verschiedene Schirm-Gegenstand- Abstände
$e$ bei scharfem Bild die Bild- und Gegenstandweite notiert. Dies wird
für eine weitere Linse wiederholt

\noindent Anschließend wird die Methode nach Bessel zur Bestimmung der Brennweite
verwendet. Bei festem Abstand $e$ zwischen Schirm und Gegenstand werden
die beiden Linsenpositionen bestimmt, bei denenn das Bild scharf ist.
Gemessen werden Gegenstandsweite $g_1$ und $g_2$ sowie Bildweite
$b_1$ und $b_2$. Die Messung wird für insgesamt zehn unterschiedliche
Abstände $e$ durchgeführt. Nun wird die chromatische Abberation mit Hilfe
eines blauen und eines roten Filters untersucht. Für jeweils fünf Abstände
$e$ pro Filter werden unter der Methode von Bessel erneut Gegenstands- und
Bildweiten gemessen.

\noindent Zuletzt wird ein Linsensystem aus konkaver und konvexer Linse
mit der Methode nach Abbe untersucht.
Es wird darauf geachtet, dass die Linsen betraglich die gleiche
Brennweite besitzen. Als Referenzpunkt $A$ wird die Mitte zwischen den
Linsen gewählt. Der Abbildungsmaßstab, also die Bild- und Gegenstandsgröße,
sowie die Bild- und Gegenstandsweiten werden gemessen. Dies wird insgesamt
für zehn Gegenstandsweiten durchgeführt.
