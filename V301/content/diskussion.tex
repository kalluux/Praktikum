\section{Diskussion}
\label{sec:Diskussion}

Der theoretische Wert für die Leerlaufspannung beträgt
\begin{align*}
U_0 = 1,4 \si{\volt} ,
\end{align*}
\noindent und der experimentell bestimmte Wert beträgt
\begin{align*}
U_0 &= (1,46 \pm 0,03)\si{\volt} .
\end{align*}
Die prozentuale Abweichung der beiden Werte beträgt ca. 4,3 Prozent.
Da außerdem der Fehler der Leerlaufspannung
\begin{align*}
\Delta U_0 = (7,8 \pm 0,7) \cdot 10^{-7} \: \symup{V} 
\end{align*}
sehr klein ist, können größere systematische oder statistische Fehlerquellen ausgeschlossen werden und die Messung als relativ genau beurteilt werden.
Die zweite berechnete Leerlaufspannung beträgt
\begin{align*}
U_0 =  (1,43 \pm 0,01)\si{\volt} ,
\end{align*}
und die Abweichung liegt bei ca. 2,1 Prozent. Das oben gesagte kann also auch auf diese Messreihe übertragen werden.
Für die restlichen Leerlaufspannungen und Innenwiderstände gibt es keine Theoriewerte, jedoch kann man anhand der Graphen sagen,
dass die Messung genau verlief. Die Messwerte liegen nämlich ziemlich nah an bzw. auf der Ausgleichsgeraden.

\noindent Auch die Werte der Leistung liegen in gutem Bereich, da sie bis auf wenige Ausnahmen nah an der Kurve liegen. 





